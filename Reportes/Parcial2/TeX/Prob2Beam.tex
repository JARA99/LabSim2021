\documentclass{beamer}
% \usetheme[numbering = none]{metropolis}%Usemetropolistheme


\title{Segundo examen parcial}
\date{4 de mayo de 2021}
\author{Jorge Alejandro Rodriguez Aldana}
\institute{Escuela de Ciencias físicas y matemáticas}
\begin{document}
\maketitle
\section{Problema 2}

\begin{frame}
    \frametitle{Metodología}

    Para realizar este problema utilicé el método de bisección, por que, aunque se ve más hambriento de poder, también es más simple, lo que evita código más complejo, y con esto, la documentación también es más sencilla de hacer.   
    
    \onslide<2>{\begin{block}{Plan}
        Comenzar realizando la gráfica para determinar el intervalo inicial a darle al programa. Y después proceder a correr el programa con este intervalo.
    \end{block}}

\end{frame}

\begin{frame}
    \frametitle{Programación}

    \begin{block}{Funciones principales}
        \begin{itemize}[<+- | alert@+>]
            \item main(): Ejecuta iteraciones hasta minimizar el error a 1\%. Despues imprime la raiz
            \item eval(): Calcula el punto medio del intervalo actual, y devuelve un nuevo intervalo
            \item err(): Calcula el error del punto medio actual
        \end{itemize}
    \end{block}


    \begin{block}{Funciones secundarias}
        \begin{itemize}[<+- | alert@+>]
            \item f(x): Función que emula la función matemática propuesta
        \end{itemize}
    \end{block}

\end{frame}

\begin{frame}
    \frametitle{Automatización}

    A diferencia del problema 1, aquí no había mucho que automatizar, así que la automatización es solamente un script de bash que compila y ejecuta el código en C, y la gráfica en gnuplot, y después de imprimir la raíz en la terminal, abre el pdf con la gráfica (válido solo para gnome).

\end{frame}

{\usebackgroundtemplate{%
  \includegraphics[width=\paperwidth,height=\paperheight]{Graphics/jawa.png}} 
\begin{frame}

\end{frame}}
\end{document}