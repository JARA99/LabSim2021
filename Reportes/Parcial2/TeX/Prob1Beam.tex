\documentclass{beamer}
% \usetheme[numbering = none]{metropolis}%Usemetropolistheme


% \usepackage{xcolor}
% \usepackage{listings}

% \definecolor{mGreen}{rgb}{0,0.6,0}
% \definecolor{mGray}{rgb}{0.5,0.5,0.5}
% \definecolor{mPurple}{rgb}{0.58,0,0.82}
% \definecolor{backgroundColour}{rgb}{0.95,0.95,0.92}

% \lstdefinestyle{CStyle}{
%     backgroundcolor=\color{backgroundColour},   
%     commentstyle=\color{mGreen},
%     keywordstyle=\color{magenta},
%     numberstyle=\tiny\color{mGray},
%     stringstyle=\color{mPurple},
%     basicstyle=\footnotesize,
%     breakatwhitespace=false,         
%     breaklines=true,                 
%     captionpos=b,                    
%     keepspaces=true,                 
%     numbers=left,                    
%     numbersep=5pt,                  
%     showspaces=false,                
%     showstringspaces=false,
%     showtabs=false,                  
%     tabsize=2,
%     language=C
% }


\title{Segundo examen parcial}
\date{4 de mayo de 2021}
\author{Jorge Alejandro Rodriguez Aldana}
\institute{Escuela de Ciencias físicas y matemáticas}
\begin{document}
\maketitle
\section{Problema 1}
\begin{frame}{Planteamiento}
Dada una tabla de datos de velocidad y tiempo de un objeto se nos pedía:

 \begin{itemize}[<+- | alert@+>]
    \item Una gráfica que comparara los valores
    \begin{itemize}
        \item Tabular los datos en un archivo de texto, y agregar una columna con el error de tiempo.
        \item Plotear los datos con Gnuplot
    \end{itemize}
    \item Un ajuste de una recta que mejor se aproximara 
    \begin{itemize}
        \item Agregar una función a Gnuplot y hacerle un \textit{fit}
        \item Programar el método de mínimos cuadrados
    \end{itemize}
    \item Obtener la aceleración aproximada
    \begin{itemize}
        \item La aceleración es igual a la pendiente estimada
    \end{itemize}
    \item Estimar la velocidad en $t=15s$
    \begin{itemize}
        \item La velocidad dependiente del tiempo está dada por la ecuación de la recta, solo hace falta valuarla en $t=15s$
    \end{itemize}
  \end{itemize}
\end{frame}

\begin{frame}
    \frametitle{Mínimos cuadrados}

    Se busca aproximar las constantes $m$ y $b$ para la ecuación de la recta $y(x)=mx+b$
    
    \bigskip

    \onslide<2->{\alert<2>{\begin{equation}
        m=\frac{n\sum_{k=1}^n\left(x_ky_k\right)-\sum_{k=1}^nx_k\sum_{k=1}^ny_k}{n\sum_{k=1}^nx_k^2-\left(\sum_{k=1}^nx_k\right)^2}
    \end{equation}}}

    \onslide<3->{\alert<3>{\begin{equation}
        b=\frac{n\sum_{k=1}^ny_k-m\sum_{k=1}^nx_k}{n}
    \end{equation}}}
\end{frame}

\begin{frame}
    \frametitle{Programación mínimos cuadrados}


    % \begin{alertblock}{Main}
    %     \begin{lstlisting}[style=CStyle]
    %         void main(){
    %             double slope = m();
    %             printf("g(x) = (%f*x)+(%f)\n",slope,b(slope));
    %         }
    %     \end{lstlisting}
    % \end{alertblock}

    % \only<2>{\begin{alertblock}{Main}
    %     \begin{lstlisting}[style=CStyle]
    %         void main(){
    %             double slope = m();
    %             printf("g(x) = (%f*x)+(%f)\n",slope,b(slope));
    %         }
    %     \end{lstlisting}
    % \end{alertblock}}
    
    \begin{block}{Funciones principales}
        \begin{itemize}[<+- | alert@+>]
            \item main(): Imprime los return de las funciones m() y b(m)
            \item m(): Calcula y entrega la pendiente con el método de mínimos cuadrados
            \item b(m): Calcula y entrega el desplazamiento de la recta, en otras palabras, la velocidad inicial
        \end{itemize}
    \end{block}


    \begin{block}{Funciones secundarias}
        \begin{itemize}[<+- | alert@+>]
            \item SumDProd(a,b): Dadas dos listas de $n$ elementos, calcula el producto en i-esimo elemento de cada lista, y luego suma los productos
            \item ProdDSum(a,b): Dadas dos listas de $n$ elementos, calcula la suma de cada lista y luego realiza el producto de estas sumas 
            \item Sum(a): Dada una lista, realiza la suma de los elementos en esta
        \end{itemize}
    \end{block}

\end{frame}

\begin{frame}
    \frametitle{Errores}

    El proceso para calcular los errores de $m$ y $b$, $\Delta m$ y $\Delta b$ respectivamente, fue uno muy similar. Se realizaron las funciones $Dm()$ y $Db()$ que realizaban el cálculo matemático.

\end{frame}

\begin{frame}
    \frametitle{Automatización}

    Para automatizar todo el proceso programé un código simple en Bash que realiza las siguientes acciones:

    \begin{itemize}[<+- | alert@+>]
        \item Compila los archivos de C
        \item Almacena la salida en una variable
        \item Duplica un archivo .gp pre programado
        \item Agrega la función de la recta a este archivo
        \item Compila la gráfica
        \item Imprime los valores de la aceleración y la velocidad a los 15s
        \item Abre el pdf (válido solo para gnome)
    \end{itemize}

\end{frame}

{\usebackgroundtemplate{%
  \includegraphics[width=\paperwidth,height=\paperheight]{Graphics/jawa.png}} 
\begin{frame}

\end{frame}}
\end{document}